% Преамбула

% oneside - одностороняя печать (есть ещё twoside)
% final   - финальная версия документа (есть ещё draft)
% 14pt    - размер шрифта 14-й кегль
\documentclass[oneside,final,12pt]{book}

\usepackage[T2A]{fontenc}

% В книге говориться, что можно использовать koi8-r (для Linux)
% и cp1251 для Windows
% Я выбрал utf8. .tex файлы сохраняю в этой кодировке. Работает на Linux и Windows.
\usepackage[utf8]{inputenc}

% Пакет babel адаптирует LATEX к работе с русским языком
\usepackage[russianb]{babel}

% Настраивание размера полосы набора BEGIN======================================
\usepackage{vmargin}
\setpapersize{A4}

% Значение параметров:
% 1     - левое поле
% 2     - верхнее поле
% 3     - правое поле
% 4     - нижнее поле
% 5,6,7 - для управления верхним и нижним колонтитулами
% 8     - расположение по вертикали номреа страницы. Расстояние между нижним краем нижней строки и нижним краем номера страницы
\setmarginsrb{2cm}{1.5cm}{1cm}{1.5cm}{0pt}{0mm}{0pt}{13mm}
% Настраивание размера полосы набора END========================================

\usepackage{epigraph}

% Красная строка
\usepackage{indentfirst}

% Предотвращшение залезания строк на поля, даже если для этого требуется заполнить строку недопустимо длинными пробелами
\sloppy

% Для вставки заранее подготовленных рисунков
\usepackage{graphicx}

% Листинги программ
\usepackage{verbatim}
\usepackage{moreverb}
\usepackage{listings}

% Математические формулы
\usepackage{amsmath}
\usepackage{amsfonts}

% Для устранения ворнингов про \circle, \oval...
\usepackage{pict2e}


\begin{document}

\noindent
Привет!

\bigskip
\noindent
Список специальных символов:
\\
\~             \ \
\#             \ \
\$             \ \
\%             \ \
\^             \ \
\&             \ \
\_             \ \
\{             \ \
\}             \ \
\textbackslash \ \
\" \


\noindent
\textbackslash \ \ (обратная косая черта) --- команда LATEX \\
\% --- начало комментария \\
\~ \ --- "<неразрывный пробел"> (т.е. такой пробельный символ, который недопустимо растягивать, сжимать и разрывать концом строки) \\
\textbackslash, --- неразрывный пробел уменьшенного размера

\noindent
{\bf жирный текст} {\it курсивный текст}

\noindent
Некоторые обозначение для спецсимволов:

\noindent
\textasciitilde \ \
\dq             \ \
\textasciicircum

\noindent
Кавычки  \" \ в пакете {\it babel} являются спецсимволом: \\
"< --- левая  елочка                                      \\
"> --- правая елочка                                      \\
"` --- открывающая лапка (обратный апостроф)              \\
"' --- закрывающая лапка (прямой апостроф)

\noindent
"<Прочитай свежие "`Ведомости"'">~--- сказал папа

\noindent
\S

\noindent
Набор диакритических знаков:
\"a   \ \
\~a   \ \
\^a   \ \
\'a   \ \
\u{a} \ \
\r{a} \ \
\d{a} \ \
\v{a} \ \
\b{a} \ \
\c{a} \ \
\H{a}

\noindent
Дефисы и тире:                                                  \\
1) дефис: кусочно-непрерывный                                   \\
2) короткое тире: от--до, дорога отнимает тридцать--сорок минут \\
3) длинное тире --- знак препинания

В кофе можно добавить десять-пятнадцать мл. виски и сливки~--- это называется "<кофе по-ирландски">

\noindent
{\bfseries О командах.} Они могут принимать на вход параметры. Обязательные параметры заключаются в фигурные скобки, а необязательные - в квадратные.  Пример команды: \\
\"{a} \\
В некоторых случаях (например, если параметр состоит из одного символа) без фигурных скобок можно обойтись: \\
\"a \\
При использовании команд, имена которых состоят из букв, LATEX воспринимает в качестве имени команды все буквы, стоящие после обратной косой черты до первого небуквенного символа. \\
\noindent
\S 6 и \S \ 6 тоже \S{} 6
\medskip

\noindent
\{ это группа \}\\
Глобальные команды --- это команды, которые сохраняют свое действие за пределами группы, в которой они были употреблены. Обычно это особые команды.
\medskip

\noindent
{\bfseries Окружение} --- это фрагмент текста, заключенный между командами {\bfseries \textbackslash begin\{\}} и {\bfseries \textbackslash end\{\}}\\
Обе команды в качестве параметра принимают {\it имя окружения}. Окружение позволяет распространить то или иное свойство на весь заключенный в окружении текст. \\
Часть файла, находящаяся внутри окружения, образует группу.\\
Пример:\\
\begin{large}
    Large environment
\end{large}
\medskip

\noindent
{\bfseries Об абзацах} \\
\textbackslash noindent --- отмена отступа (красной строки) для абзаца. Команда вставляется непосредственно перед абзацем
\textbackslash \textbackslash --- принудительное завершение строки, но не абзаца \\
Пустая строка или команда \textbackslash par --- отделение одного абзаца от другого\\
\textbackslash bigskip \
\textbackslash medskip \
\textbackslash smallskip --- вставка между абзацами вертикального интервала
\pagebreak

\bigskip
\noindent
{\large Декларирующие команды:}
\medskip

\noindent
Font Family (гарнитуры):                                                   \\
\textbackslash rmfamily --- {\rmfamily Roman (обычная гарнитура)}          \\
\textbackslash sffamily --- {\sffamily Sans Serif (гарнитура без засечек)} \\
\textbackslash ttfamily --- {\ttfamily Typewriter (моноширинная гарнитура)}
\medskip

\noindent
Series (жирность):\\
\textbackslash bfseries --- {\bfseries bold face (включение ожирнения)}  \\
\textbackslash mdseries --- {\mdseries medium density (отмена ожирнения)}
\medskip

\noindent
Форма (shape):                                                    \\
\textbackslash itshape --- {\itshape italics (курсив)}            \\
\textbackslash slshape --- {\slshape slanted (наклонный шрифт)}   \\
\textbackslash scshape --- {\scshape small caps (капитель)}       \\
\textbackslash upshape --- {\upshape upright (обычное начертание)}\\
\textbackslash normalfont --- {\normalfont основной шрифт документа (устанавливает обычную гарнитуру, обычное начертание и выключает ожирнение)}\\
Область действия декларирующих команд не может распространяться дальше "<блока">, в котором они появились; в качестве такого блока, например, может выступать текст, заключенный в фигурные скобки.
\medskip

\noindent
Команды с параметром. Общий вид {\ttfamily \textbackslash textXXX\{<текст>\}} :\\
\textrm{\textbackslash textrm\{\} --- \textbackslash rmfamily}\\
\textsf{\textbackslash textsf\{\} --- \textbackslash sffamily}\\
\texttt{\textbackslash texttt\{\} --- \textbackslash ttfamily}
\medskip

\noindent
\textbf{\textbackslash textbf\{\} --- \textbackslash bfseries}\\
\textmd{\textbackslash textmd\{\} --- \textbackslash mdseries}
\medskip

\noindent
\textit{\textbackslash textit\{\} --- \textbackslash itshape}\\
\textsl{\textbackslash textsl\{\} --- \textbackslash slshape}\\
\textsc{\textbackslash textsc\{\} --- \textbackslash scshape}\\
\textup{\textbackslash textup\{\} --- \textbackslash upshape}
\medskip

\noindent
\textnormal{\textbackslash textnormal\{\} --- \textbackslash normalfont} \\
\emph{\textbackslash emph --- (emphasis -- выделение) переключение между обычным шрифтом и курсивом. Декларирующей формы у этой команды нет.}
\medskip

\noindent
Команды в форме окружения:

\begin{bfseries}
    \begin{scshape}
        \noindent
        Жирный капитель.
    \end{scshape}
\end{bfseries}
\medskip

\noindent
Устаревшая форма декларирующих команд:\\
\textbackslash rm \
\textbackslash sf \
\textbackslash tt \ \ \ \
\textbackslash bf \ \ \ \
\textbackslash it \
\textbackslash sl \
\textbackslash sc
\medskip

\noindent
Команды, задающие размер шрифта:

\noindent
{\tiny \textbackslash tiny --- самый мелкий возможный размер, обычно совершенно нечитаемый}
\medskip

\noindent
{\scriptsize \textbackslash scriptsize --- чуть больше, но все равно очень мелкий}
\medskip

\noindent
{\footnotesize \textbackslash footnotesize --- размер, которым набираются сноски}
\medskip

\noindent
{\small \textbackslash small --- шрифт чуть меньше обычного}
\medskip

\noindent
{\normalsize \textbackslash normalsize --- обычный размер шрифта (тот, что задан в преамбуле)}
\medskip

\noindent
{\large \textbackslash large --- слегка укрупненный}
\medskip

\noindent
{\Large \textbackslash Large --- ещё крупнее}
\medskip

\noindent
{\LARGE \textbackslash LARGE --- совсем купный}
\medskip

\noindent
{\huge \textbackslash huge --- гигантский}
\medskip

\noindent
{\Huge \textbackslash Huge --- супергигантский}
\medskip

\noindent
\textbf{Интерлиньяж - это расстояние между базовыми линиями строк.}
\medskip

\noindent
Интерлиньяж определяется в тот момент, когда очередная строка завершается. Либо по принудительной команде либо по окончании верстки целого абзаца.
\medskip

\noindent
\textbf{Всегда набирайте абзац шрифтом одного размера и всегда оставляйте пустую строку перед командой смены шрифта}

\clearpage

\noindent
\textbf{
Рубрикация --- это разбивка документа на части, главы, параграфы и т.п.}
\medskip

\noindent
Возможные разделы:\\
\textbackslash part --- часть\\
\textbackslash chapter --- глава (для класса документа \textit{article} не работает)\\
\textbackslash section --- секция (типа параграф, но это слово в командах будет дальше)\\
\textbackslash subsection --- подсекция\\
\textbackslash subsubsection --- подподсекция\\
\textbackslash paragraph --- пункт (типа абзац)\\
\textbackslash subparagraph --- подпункт\\
Все эти команды имею один параметр -- заголовок соответствующего раздела.
\medskip

\noindent
\textbackslash tableofcontents --- формирование оглавления\\
\textbf{Внимание!} Учтите, что теперь вам может понадобиться каждый раз прогонять программу \texttt{latex} дважды. Дело в том, что для формирование оглавления LATEX использует дополнительный файл (с расширением \texttt{.toc}), в который во время трансляции вашего исходного текста заносится информация для включения в оглавление (особенно если оглавление вставляется в начало документа), поскольку файл \texttt{.toc} может отсутствовать или содержать устаревшую информацию. Вот пример:
\medskip

\tableofcontents

\part{Part}
Здесь текст части.
%\chapter{Глава}
\section{Section}
Здесь текст секции.
\subsection{Subsection}
Здесь текст подсекции.
\subsubsection{Subsubsection}
Здесь текст подподсекции.
\paragraph{Paragraph}
Здесь текст пункта.
\subparagraph{Subparagraph}
Здесь текст подпункта.
\medskip

\noindent
Перекрестные ссылки:\\
\textbackslash label\{<метка>\} --- установить метку. В качестве имени метки можно использовать любую последовательность латинских букв, цифр и некоторых знаков препинания (особенно часто используется двоеточие, подчеркивание и тире), например: \textbackslash label \{main:theorem:prove\}
\medskip

\noindent
\textbackslash ref \{<метка>\} --- ссылка на метку\\
Пример: \texttt{
Доказательство см.~в~\textbackslash S \textbackslash ,\textbackslash ref\{main:theorem:prove\}}
\medskip

\noindent
\textbackslash pagerf\{<метка>\} --- ссылка на страницу\\
Пример: \texttt{
приведено на стр.\textbackslash ,\textbackslash pageref\{main:theorem:prove\}}
\medskip

\noindent
Приложения:\\
\textbackslash appendix --- команда отделения текста всех приложений от основного текста документа. Дается только один раз. Экспериментально установлено, что для класса документа \textit{article} каждое приложение оформляется командой \textbackslash section и последующими. Приложения так же будут включены в оглавление.\\
\medskip

\noindent
Низкоуровневое управление\\
"<звёздочные версии команд"> --- это команды со звёздочками: \textbackslash part*, \textbackslash section*, \textbackslash subsection*... Они производят вставку сущности без номера и не влияющую на дальнейшую нумерацию.\\
\medskip

\noindent
\textbackslash addcontentsline \{toc\}\{part\}\{<имя в оглавлении>\} --- принудительное добавление ненумерованной сущности в оглавление. Например,\\
\medskip

\noindent
\texttt{
\textbackslash section* \{Литерура\}\\
\textbackslash addcontentsline\{toc\}\{part\}\{Литература\}}\\
\medskip

\noindent
\section*{Литература}
\addcontentsline{toc}{part}{Литература}

\clearpage

\noindent
Окружение:\\
\medskip

\noindent
\texttt{
\textbackslash begin(thebibliography)\{00\}\\
\textbackslash bibitem[<номер>]\{<метка>\}\\
\textbackslash end(thebibliography)\\
}
\medskip

\noindent
<номер> --- необязательный параметр\\
\textbackslash \{00\} --- используется для указания максимальной ширины номера ссылки и используется LATEX'ом для горизонтального выравнивания элементов списка. Если ваш список литературы состоит не более чем из девяти пунктов, можно вместо \texttt{00} указать просто \texttt{0}, если же число источников составляет сто и больше, стоит указать \texttt{000}.\\
\textbackslash cite --- по этой команде ссылаемся на источник из списка литературы.

\begin{thebibliography}{00}

\bibitem{knut:tex} Дональд Е.~Кнут.
\emph{Всё про \TeX}. Изд-во AO RDTeX,
Протвино, 1993.

\bibitem{kuhn:revolutions} T.~S.~Kuhn.
\emph{The structure of scientific revolutions}.
University of Chicago Press,
Chicagom second, enlarged edition, 1970.

\bibitem{stolyarov:oorefal} А.~В.~Столяров.
\emph{Расширенный функциональный аналог языка Рефал для мультипарадигмального программирования.}
//Л.~Н.~Королев, ред., \emph{Программные системы и инструменты}. Тематический сборник, том~2, стр.~184--195. Издательский отдел ВМиК МГУ, Москва, 2001.

\end{thebibliography}

\clearpage

\noindent
Новая команда:\\
\texttt{
\textbackslash newcommand\{<имя>\}[<арг>][<умолч>]{<содержание>}}\\
Здесь имя команды берется в фигурные скобки в отличие от книги. Наверное фигурные скобки -- это по-новому. Хотя без них тоже работает. Пример:\\
\texttt{
\textbackslash newcommand\textbackslash notion[1]\{textit\{\#1\}\}}
\medskip

\newcommand{\notion}[1]{\textit{#1}}

\noindent
А вот и \notion{пример!}.\\
Еще один пример:\\
\newcommand{\Brief}[1][*]{
    \bigskip
    \centerline{#1~~#1~~#1~~#1~~#1}
    \bigskip
}
\Brief\\
\Brief[+]
\medskip

\noindent
Новое окружение:\\
\texttt{
\textbackslash newenvironment\{<имя>\}[<арг>][<умолч>]\{<нач>\}\{<кон>\}}\\
Пример:\\
\texttt{
\textbackslash newenvironment\{remarks\}\{\\
    \textbackslash begin\{sffamily\}\\
    \textbackslash small\\
\}\{\\
    \textbackslash par\\
    \textbackslash normalsize\\
    \textbackslash end\{sffamily\}
\}
}

\newenvironment{remarks}{
    \begin{sffamily}
    \small
}{
    \par
    \normalsize
    \end{sffamily}
}

\begin{remarks}
    Remarks environment in action.
\end{remarks}

\clearpage

\noindent
Ненумерованное перечисление:\\
\texttt{
\textbackslash begin\{itemize\}...\textbackslash end\{itemize\}}
Внутри окружения используется команда \texttt{\textbackslash item} (обязательно хотя бы одна). Пример:\\
\texttt{
\textbackslash begin\{itemize\}\\
\phantom{~~~~} \textbackslash item логику изложения;\\
\phantom{~~~~} \textbackslash item полноту охвата материала;\\
\phantom{~~~~} \textbackslash item адекватность используемых терминов;\\
\phantom{~~~~} \textbackslash item орфографическую грамотность текста.\\
\textbackslash end\{itemize\}}\\
\medskip

\noindent
необходимо обращать внимание на:
\begin{itemize}
    \item логику изложения;
    \item полноту охвата материала;
    \item адекватность используемых терминов;
    \item орфографическую грамотность текста.
\end{itemize}

\noindent
Допускаются вложенности:
\begin{itemize}
    \item растения, в том числе
    \begin{itemize}
        \item одноклеточные растения,
        \item травы,
        \item деревья, среди которых можно назвать
        \begin{itemize}
            \item сосны,
            \item березы,
            \item клены и т.п.,
        \end{itemize}
        \item кустарники;
    \end{itemize}
    \item животные, включая
    \begin{itemize}
        \item простейших,
        \item рыб,
        \item земноводных,
        \item птиц,
        \item млекопитающих;
    \end{itemize}
    \item грибы, которые не относятся ни к растениям, ни к животным
\end{itemize}


\noindent
Нумерованное перечисление:\\
\texttt{
\textbackslash begin\{enumerate\}...\textbackslash end\{enumerate\}}

\clearpage

\noindent
\textbf{
Плавающий объект --- это нечто, имеющее заголовок, номер и содержимое объекта.}\\
\medskip

\noindent
Это делается для автоматизации такого текста как: \texttt{см.~табл.\,2.7 на стр.\,30} и \texttt{см.~рис.\,1.2 на стр.\,8}\\
\medskip

\noindent
\texttt{
\textbackslash \{table\}\\
\phantom{~~~~}\% ... содержание рисунка ...\\
\phantom{~~~~}\textbackslash caption\{Удельный вес некоторых веществ\}\\
\phantom{~~~~}\textbackslash label\{<метка>\}\\
\textbackslash \{table\}}\\
\medskip

\noindent
\texttt{
\textbackslash \{figure\}\\
\phantom{~~~~}\% ... содержание рисунка ...\\
\phantom{~~~~}\textbackslash caption\{Удельный вес некоторых веществ\}\\
\phantom{~~~~}\textbackslash label\{<метка>\}\\
\textbackslash \{figure\}}\\
\medskip

\noindent
\texttt{
см.~табл.\textbackslash ,\textbackslash ref\{some\textbackslash \_densities\} на стр.\textbackslash , \textbackslash pageref\{some\textbackslash \_densities\}}\\
\medskip

\noindent
Необязательные параметры в окружениях \texttt{table} и \texttt{figure}:\\
Предписание разместить плавающий объект:\\
\texttt{t} --- (top) в верхней части страницы;\\
\texttt{b} --- (bottom) в нижней части страницы;\\
\texttt{p} --- (page) на отдельной странице;\\
\texttt{h} --- (here) сразу же после появления в исходном тексте.\\
\medskip

\noindent
\texttt{
\textbackslash \{figure\}[hp]} --- означает, что данный рисунок хотелось бы увидеть размещенным непосредственно в том месте текста, где он описан, если же это не получится --- то вынести его на отдельную страницу.\\
\textbf{
\LaTeX \ предпочитает размещать объекты в верхней части страницы.}

\noindent
\begin{verbatim}
\begin{table}[h]
    \begin{tabular}{|r|c|p{0.15\textwidth}|p{0.25\textwidth}|c|r|}
        \hline No. & Тип & Автор & Заглавие & Год & Тир. \\ \hline
        1 & книга & Артур Конан Дойл & Собака Баскервилей
        & 1975 & 10\,000 \\ \hline
        2 & книга & Жюль Верн & Пять недель на воздушном шаре
        & 1981 & 7000 \\ \hline
        3 & журнал & \multicolumn{2}{c|}{Вокруг света No. 5}
        & 1995 & 5000 \\ \hline
    \end{tabular}
    \caption{Пример таблицы}
    \label{table_example}
\end{table}
\end{verbatim}

\noindent
\begin{table}[ht]
\begin{tabular}{|r|c|p{0.15\textwidth}|p{0.25\textwidth}|c|r|}
    \hline No. & Тип & Автор & Заглавие & Год & Тир.\\ \hline
    1 & книга & \raggedright{Артур Конан Дойл} & Собака Баскервилей
      & 1975 & 10\,000\\ \hline
    2 & книга & Жюль Верн & Пять недель на воздушном шаре
      & 1981 & 7000\\ \hline
    3 & журнал & \multicolumn{2}{c|}{Вокруг света No. 5}
      & 1995 & 5000\\ \hline
\end{tabular}
\caption{Пример таблицы}
\label{table_example}
\end{table}
\medskip

\noindent
\texttt{
\textbackslash begin\{tabular\}...\textbackslash end\{tabular\}} --- окружение для таблицы\\
\texttt{l} --- left\\
\texttt{r} --- right\\
\texttt{c} --- center\\
\texttt{p\{<ширина>\}} --- часто бывает удобно привязать ширину столбца к общей ширине наборе. Это можно сделать примерно так \texttt{p\{0.7\textbackslash textwidth\}}. Но также можно задать в сантиметрах \texttt{cm}, миллиметрах \texttt{mm} или пунктах \texttt{pt} и прочее. Рисунки:

\begin{verbatim}
\begin{figure}
    \begin{picture}(60, 100)(0, 0)
        \put(5,  50){\line( 1, 2){25}}
        \put(5,  50){\line( 1,-2){25}}
        \put(5,  50){\line( 1, 0){50}}
        \put(55, 50){\line(-1, 2){25}}
        \put(55, 50){\line(-1,-2){25}}
        \put(30,  0){\line( 0, 1){100}}
    \end{picture}
    \caption{Рисунок из отрезков в \LaTeX-графике}
    \label{fig_example}
\end{figure}
\end{verbatim}

\noindent
\begin{figure}[ht]
\begin{picture}(60, 100)(0, 0)
    \put(5,  50){\line( 1, 2){25}}
    \put(5,  50){\line( 1,-2){25}}
    \put(5,  50){\line( 1, 0){50}}
    \put(55, 50){\line(-1, 2){25}}
    \put(55, 50){\line(-1,-2){25}}
    \put(30,  0){\line( 0, 1){100}}
\end{picture}
\caption{Рисунок из отрезков в \LaTeX-графике}
\label{fig_example}
\end{figure}

\noindent
\texttt{
\textbackslash begin\{picture\}(<ширина>, <высота>)(<сдвиг по х>,<сдвиг по y>)}\\
Начало координат в нижнем левом углу.\\
\medskip

\noindent
Рисунок состоит из элементов, каждый из которых задается командой:\\
\texttt{
\textbackslash put(<x>,<y>){<описанние>}}
\begin{verbatim}
\put(40, 60){просто текст}
\end{verbatim}

\begin{picture}(100, 100)(0, 0)
    \put(40, 60){просто текст}
\end{picture}

\noindent
\begin{picture}(200, 350)(0, 0)
    % начало координат
    \put(0, 0){O}

    % ось ОХ
    \put(0, 0){\vector(1, 0){200}}
    \put(200, 0){x}

    % ось OY
    \put(0, 0){\vector(0, 1){350}}
    \put(0, 350){y}

    \put(45, 45){A}

    \put(90, 50){\line(0, 1){300}}

    % x1 = 50, x2 = 50, dx = 1, dy = 0, p = 40
    % tg a = dy / dx = 0 / 1 = 0, a = arctg(0) = 0
    % x2 = x1 + p = 50 + 40 = 90
    % y2 = y1 + p * tg(a) = 50 + 40 * 0 = 50
    \put(50, 50){\line(1, 0){40}} % (90, 50)
    \put(94, 50){(1,0)}
    %=======================================================

    % x1 = 50, x2 = 50, dx = 1, dy = 1, p = 40
    % tg a = dy / dx = 1 / 1 = 0, a = arctg(0) = 45deg
    % x2 = x1 + p = 50 + 40 = 90
    % y2 = y1 + p * tg(a) = 50 + 40 * 1 = 90
    \put(50, 50){\line(1, 1){40}} % (90, 90)
    \put(94, 90){(1,1)}


    \put(50, 50){\line(1, 2){40}} % (90, 130)
    \put(94, 130){(1, 2)}

    \put(50, 50){\line(1, 3){40}} % (90, 170)
    \put(50, 50){\line(1, 4){40}} % (90, 210)
    \put(50, 50){\line(1, 5){40}} % (90, 250)
    \put(50, 50){\line(1, 6){40}} % (90, 290)

    \put(50, 50){\line(2, 1){40}}
    \put(50, 50){\line(2, 3){40}}
    \put(50, 50){\line(2, 5){40}}

    \put(50, 50){\line(3, 1){40}}
    \put(50, 50){\line(3, 2){40}}
    \put(50, 50){\line(3, 4){40}}
    \put(50, 50){\line(3, 5){40}}

    \put(50, 50){\line(4, 1){40}}
    \put(50, 50){\line(4, 3){40}}
    \put(50, 50){\line(4, 5){40}}

    \put(50, 50){\line(5, 1){40}}
    \put(50, 50){\line(5, 2){40}}
    \put(50, 50){\line(5, 3){40}}
    \put(50, 50){\line(5, 4){40}}
    \put(50, 50){\line(5, 6){40}}

    \put(50, 50){\line(6, 1){40}}
    \put(50, 50){\line(6, 5){40}}
\end{picture}
\medskip

\noindent
\texttt{
\textbackslash put(x1, y1)\{\textbackslash line(dx, dy)\{p = проекция\_на\_ось\_OX\}} --- $\tg(\alpha) = \frac{dy}{dx}$\\
dx, dy --- целые числа от ---6...6\\
если~$dx=0$,~тогда~$x_2 = x_1$, $y_2 = p$,\\
иначе~$x_2= x_1 + p$,~$y_2 = y_1 + p \tg(\alpha)$\\
\begin{picture}(110, 110)(-10, -10)
    \put(0,0){\vector(0, 1){100}}
    \put(0,0){\vector(1, 0){100}}
    \put(-7, -7){0}
    \put(95, -7){$x$}
    \put(-7, 95){$y$}
    \put(30, 30){\oval(20, 30)}
    \put(50, 50){\circle{50}}
    \put(70, 50){\circle*{15}}
    \put(60, 90){\frame{~sample~}}
\end{picture}
\medskip

\noindent
\texttt{\textbackslash vector} --- аналогична команде \texttt{\textbackslash line}, только на конце рисует стрелку\\
\texttt{\textbackslash circle\{<диаметр>\}} --- рисует окружность\\
\texttt{\textbackslash circle*\{<диаметр>\}} --- рисует круг (закрашенную окружность)\\
\texttt{\textbackslash oval\{x, y\}} --- рисует овал\\
\texttt{\textbackslash frame\{<аргумент>\}} --- заключает аргумент в рамку\\
\medskip

\noindent
Вставка заранее подготовленного рисунка:
%\begin{figure}[h]
%    \centering
%    \includegraphics[width=0.9\textwidth]{xfig_example}
%    \caption{Заранее подготовленный рисунок}
%    \label{xfig_example}
%\end{figure}

\begin{verbatim}
% В преамбуле подключить пакет:
% Для вставки заранее подготовленных рисунков
\usepackage{graphicx}

\begin{figure}[t]
    \centering
    \includegraphics[width=0.9\textwidth]{xfig_example}
    \caption{Заранее подготовленный рисунок}
    \label{xfig_example}
\end{figure}
\end{verbatim}

\noindent
\texttt{
\textbackslash includegraphics[<опции>]\{<имя файла>\}}\\
<имя файла> указывается без расширения, что позволяет использовать один и тот же исходный текст с обычным \LaTeX'ом, так и с программой \texttt{pdflatext}; в первом случае система попытается найти файл с расширением \texttt{.eps} или \texttt{.ps}, во втором --- с расширением \texttt{.png}, \texttt{.pdf}, \texttt{.jpg}, \texttt{.mps}, \texttt{.tif}.\\
Для подготовки файлов в ОС Linux используется программа XFig. Она сохраняет файлы в формате \texttt{.fig}; сконвертировать такой файл в формат \texttt{.eps} можно с помощью программы \texttt{fig2dev}:\\
\texttt{fig2dev -j -L eps myfile.fig myfile.eps}\\
\texttt{fig2dev -L pdf myfile.fig myfile.pdf}\\
\medskip

\noindent
Бывает так, что программа просмотра dvi файлов не отображает заранее подготовленные рисунки. Скорее всего это проблема не в dvi файле, а в программе просмотра. Для доказательства можно из dvi файла получить pdf. В последнем всё должно быть в порядке:\\
\texttt{
   dvipdf  file.dvi file.pdf}

\noindent
\texttt{
\textbackslash begin\{verbatim\}...\textbackslash end\{verbatim\}} --- печает текст как есть (дословно)\\
\texttt{
\textbackslash verb <разделитель>...<разделитель>} --- аналогично. Но внутренний текст не переносится на другую строку. Разделитель --- любой символ кроме латинских букв и звёздочки:
\begin{verbatim}
\verb:(a[~'$']&b):.
\end{verbatim}
\verb:(a[~'$']&b):.\\
Окружение \texttt{vebrbatim} и команда \texttt{verb} и имеют альтернативную "<звёздочную"> форму. При использовании которых символ пробела превращается в знак "<\textvisiblespace">\\
Средства из дополнительных пакетов:\\
\medskip

\noindent
Пакет \texttt{verbatim}\\
\texttt{\textbackslash verbatiminput\{<имя файла>\}}
\\
\texttt{\textbackslash verbatiminput}
\verbatiminput{figure.py}
\texttt{==============================================================}

\noindent
\texttt{\textbackslash listinginput}
\listinginput{1}{figure.py}
\texttt{==============================================================}

\noindent
\texttt{\textbackslash lstinputlisting}
\lstinputlisting[language=python, firstnumber=1, numbers=left]{figure.py}
\texttt{==============================================================}

\clearpage

\noindent
\texttt{
\textbackslash footnote\{<текст>\}}\\
Это сноска\footnote{А это текст сноски!}

\begin{verbatim}
\marginpar[<левая версия>]{текст}
\end{verbatim}

\marginpar[]{т1}

\clearpage

\noindent
Окружения:
\texttt{
math \ \ \ \ \ \ \ \ \$ ... \$ \ \ \  \textbackslash( ... \textbackslash) \\
displaymath \$\$ ... \$\$ \ \ \textbackslash[ ... \textbackslash ] \\
equation}
\\
Дополнительные математические шрифты из пакета \texttt{amsfonts}:\\
\texttt{\textbackslash mathcal\{\}} --- каллиграфический\\
\texttt{\textbackslash mathfrak\{\}} --- готический\\
\texttt{\textbackslash mathbb\{\}} --- ажурный

\end{document}
