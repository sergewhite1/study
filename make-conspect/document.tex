% Преамбула

% oneside - одностороняя печать (есть ещё twoside)
% final   - финальная версия документа (есть ещё draft)
% 14pt    - размер шрифта 14-й кегль
\documentclass[oneside,final,12pt]{book}

\usepackage[T2A]{fontenc}

% В книге говориться, что можно использовать koi8-r (для Linux)
% и cp1251 для Windows
% Я выбрал utf8. .tex файлы сохраняю в этой кодировке. Работает на Linux и Windows.
\usepackage[utf8]{inputenc}

% Пакет babel адаптирует LATEX к работе с русским языком
\usepackage[russianb]{babel}

% Настраивание размера полосы набора BEGIN======================================
\usepackage{vmargin}
\setpapersize{A4}

% Значение параметров:
% 1     - левое поле
% 2     - верхнее поле
% 3     - правое поле
% 4     - нижнее поле
% 5,6,7 - для управления верхним и нижним колонтитулами
% 8     - расположение по вертикали номреа страницы. Расстояние между нижним краем нижней строки и нижним краем номера страницы
\setmarginsrb{2cm}{1.5cm}{1cm}{1.5cm}{0pt}{0mm}{0pt}{13mm}
% Настраивание размера полосы набора END========================================

\usepackage{epigraph}

% Красная строка
\usepackage{indentfirst}

% Предотвращшение залезания строк на поля, даже если для этого требуется заполнить строку недопустимо длинными пробелами
\sloppy

% Для вставки заранее подготовленных рисунков
\usepackage{graphicx}

% Листинги программ
\usepackage{verbatim}
\usepackage{moreverb}
\usepackage{listings}

% Математические формулы
\usepackage{amsmath}
\usepackage{amsfonts}

% Для устранения ворнингов про \circle, \oval...
\usepackage{pict2e}


\begin{document}
Конспект по системе сборки make

В основном применительно к C/C++ проектам

Поэтому сперва рассмотрим сборку С/C++ проектов

\vspace{0.2cm}

\begin{tikzpicture}
	\draw [help lines] (0, 0) grid (16, 10);

	\node (c_file_0) at (0, 10) [shape=rectangle,draw]{.c};
	\node (c_file_1) at (0, 9)  [shape=rectangle,draw]{.c};
	\node (c_file_2) at (0, 8)  [shape=rectangle,draw]{.c};
	\node (preproc)  at (2, 9)  [shape=rectangle,draw,minimum height=3cm]{Preproc.};

	\draw[->] (c_file_0.east) -- (preproc);
	\draw[->] (c_file_1) -- (preproc);
	\draw[->] (c_file_2) -- (preproc);
\end{tikzpicture}

\vspace{2cm}
=============================================

Далее идут черновоки по latex'у :)

% pgfmanual, page 142

\begin{tikzpicture}[fill=blue!20]
	\draw [help lines] (0, 0) grid (6, 3);
	\draw (0, 0) -- (6, 1);

\end{tikzpicture}


% pgfmanual, page 35

We are working on
\begin{tikzpicture}
	\draw (-1.5,0) -- (1.5,0);
	\draw (0,-1.5) -- (0,1.5);
	\draw (0, 0) circle [radius=1];
\end{tikzpicture}.

\begin{tikzpicture}
	\draw [help lines] (-2, -2) grid (2, 3);
	\path ( 0,2) node [shape=circle,draw] {}
	( 0,1) node [shape=circle,draw] {}
	( 0,0) node [shape=circle,draw] {}
	( 1,1) node [shape=rectangle,draw] {}
	(-1,1) node [shape=rectangle,draw] {};
\end{tikzpicture}

\begin{tikzpicture}
	\draw [help lines] (-2, -2) grid (2, 3);
	\node (A)  at ( 0,2) [shape=circle,draw] {A};
	\node (B) at ( 0,1) [shape=circle,draw] {B};
	\node (C) at ( 0,0) [shape=circle,draw] {C};
	\node (D) at ( 1,1) [shape=rectangle,draw] {D};
	\node (E) at (-1,1) [shape=rectangle,draw] {E};

	\draw[->] (E.east) -- (B.west);
\end{tikzpicture}

\end{document}}
